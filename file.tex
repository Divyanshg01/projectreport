\documentclass[14pt ,a4paper]{extarticle}
\usepackage[top=1in , bottom=1in , left=0.5in , right=0.5in]{geometry}
\usepackage{graphicx}
\begin{document}
\tableofcontents
\pagebreak

\section{Fault And Its Classifications}
\begin{itemize}
    \item{Faults in electrical power systems are abnormal conditions that disrupt the normal flow of current. These disruptions can range from minor disturbances to catastrophic failures.}
    \item{Faults in electrical power systems can cause significant disruptions,
 including power outages, equipment damage, and safety hazards.}
\item{ To mitigate these effects, power system operators employ various protection measures such as protective relays, circuit breakers, and grounding systems.}
    \item{There are primarily open and short circuit faults}
    \item{Faults can be located using terminal fault location methods or cable screening methods}

    \item{Open Circuit fault occurs in the series of transmission line - Open Conductor Fault , 2 open conductor fault , 3 open conductor fault which causes excessive current to flow into the system}
    \item{They can be tolerated but if higher power then insulation breaks down and short circuit fault occurs}
    \item{Short Circuit Fault occurs due to insulation failure between a phase conductors and ground}
    \item{These includes}
        \begin{enumerate}
            \item{Symmetrical Faults}
            \item{Unsymmetrical Faults}
        \end{enumerate}
        \vspace{40pt}
    \subsection{Symmetrical Faults}
    \item{Arcing due to faults can lead to fire}
    \item{Voltage can fall below permissible value}
        \vspace{50pt}
        \item{Symmetrical Faults involves all three phases like}

        \begin{enumerate}
            \item{L-L-L}
            \item{L-L-L-G}
        \end{enumerate}
        \begin{center}
                \includegraphics[width=0.6\textwidth]{./assets/Two-most-common-symmetrical-fault-types.png}
            \end{center}    
    \item{Majority of symmetrical faults occur at generator terminals , system stays balance but electrical equipments can get severly damaged}
    \item{They are the most severe type of fault with highest fault current but they happens rarely}
        \subsection{Unsymmetrical Faults}
    
    \item{These fault causes unsymmetrical current , meaning variation in phase and magnitude throughout all three phases }

    \item{These faults are more frequent faults }
    \item{These includes}
        \begin{enumerate}
            
            \item{L-G} 
            \item{L-L} 
            \item{L-L-G} 
            
        \end{enumerate}
        \begin{center}
                \includegraphics[width=0.6\textwidth]{./assets/Unsymmetrical-Faults-on-Three-Power-System.jpg}
            \end{center}    
    \item{In these faults conductors make contact with other conductor or with the ground or both}
    \item{L-L faults occurs mainly due to 2 lines swinging because of high speed winds}
    \item{Here the system is unbalanced because impedance level in each phase differs , causing unbalanced current to flow between the phases}
        
\end{itemize}
\section{Machine Learning Application in Fault Analysis}
\begin{itemize}
    \item{In These applications first data is extracted out using Techniques like Discrete Wavelet Transform , fourier transform etc. Then this data is fed into the ML on basis of which it gives result. We can automate the process of fault detection with ML}
    \item{Machine Learning (ML) algorithms can be incredibly effective at identifying abnormal patterns in sensor data, which can signal potential faults or anomalies within a system.A technique by the name of Anomaly Detection can be used here.}
    \item{Machine Learning can be used for fault detection as well as fault diagnosis(FDD)}
    \item{We can use Supervised learning(with the help of labelled dataset) to classify faults into short circuit and open circuit faults}
    \item{ML models can be used to predict faults and do timely maintenance of the system using past data}
    \item{Specific location of faults can be identified with the help of ML models which can significantly reduce clearing time of the system  }
    \item{Techniques like Recurrent Neural Networks (RNNs) can analyze time-series data from sensors to detect faults based on temporal dependencies(severity of a fault is dependent on its relationship with other faults.}
    \item{Convolution Neural Networks(CNNs) can be trained on fault signatures (waveforms ,phasors ) to classify faults  }
    \item{Clustering algorithm can be used to divide fault into groups on the basis of severity}
    \item{RBFNN(Radial Basis Function Neural Network) is an Artificial Neural Network which can be used for prediction for a non-linear data set . It consists of a radial basis function used to determine faults}
    \item{ML can be used to detect faults in real time for a complex system like a Smart Grid where bi-directional flow of energy occurs , which can include Microgrids that can work with the main grid or in isolation}
\end{itemize}

\section{ Research Papers }

\subsection{Integrating discrete wavelet transform with neural networks and machine
learning for fault detection in microgrids}

\begin{itemize}
    \item{Additional difficulties in microgrid fault detection due to distributed generation specifically the bidirectional flow of energy} 
    \item{conventional systems are ineffective due to low value of fault current in MG}
    \item{Techniques of protection differs on whether the MG is connects to main grid or is working in isolation mode}
    \item{It involves generator of different capacities and types of fault current producd at various levels}
    \item{DWT extracts wavelet coefficients}

    \item{Paper suggests RBFNN for FDD of faults, DWT is used for extracting time-frequency domain characteristics}
    \item{Equation of DWT is 
            $$
            d_{j,k} = \int \limits_{- \infty}^{ \infty } s(t)\psi^*_{a,b}(t)dt = \langle{s(t), \psi_{j,k}(t) } \rangle 
            $$
        }
    \item{Usually an approximation is done to calculate them }
    \item{The decomposition level is determined by signal size and wavelet properties}
    \item{db family of wavelet is used in this}
    \item{During the learning
process of the RBFNN, new hidden units are allocated, and network
parameters are adjusted. Initially, the network starts with no hidden
units, and as training progresses, additional hidden units are added
based on the novelty of the data. }
\item{MSE(Mean Squarred Error and $R^2$ (R squared) are used to evaluate performance of models)}
    
\item{Moore-Penrose Algorithm gave the best results ,Levenberg-Marquardt, Conjugate Gradient, and Resilient-Backpropagation also showed good results,}

\item{The data for training was normalized for training the model}
\end{itemize}
\end{document}
